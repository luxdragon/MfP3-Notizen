

\setlength{\abovedisplayskip}{3pt}
\setlength{\belowdisplayskip}{3pt}
\setlength{\abovedisplayshortskip}{3pt}
\setlength{\belowdisplayshortskip}{3pt}

	\thispagestyle{empty}
	\rule{\linewidth}{1pt}
	
	\vspace{6pt}				%Die Leerzeilen müssen tatsächlich da sein, sonst funktioniert das nicht
	
	\begin{minipage}{0.6\textwidth}
		\begin{flushleft} 
		\Profs
		\end{flushleft}
	\end{minipage}
	\begin{minipage}{0.39\textwidth}
		\begin{flushright}
			Universität Hamburg
		\end{flushright}
	\end{minipage}

	\rule{\linewidth}{1pt}\\
	\begin{center}
		\Large{\textsf{\titel}}\\
		\small\textsf{Version vom \today}
\vspace{8pt}
\end{center}


\begin{figure}[htbp]
    \centering
    \includegraphics[width=.50\textwidth]{Dateien/3D_Analysis.jpg}\\
    Analysis, aber diesmal in 3D!\\
\end{figure}
\vfill
Grüßt euch, dies sind die Community MfP3-Notizen. \\

Wir erstellen sie als Nachbereitung der Vorlesung und sie dienen als eine schnelle Quelle von Definitionen und einfachen Beispielen (um sich einen Überblick zu verschaffen), wichtigen Bemerkungen aus den Übungen, sowohl als Klausurnotizen. Wir bewundern die informellen Notizen zum MfP1- und MfP2-Tutorium von Robin Löwenberg und Fabian Balzer und haben uns entschloßen mit dem gleichen Stil weiterzumachen, da es keine MfP3 und MfP4 Tutoriumnotizen mehr gibt. Die Templates wurden von Fabian erstellt und sind auf seiner \href{https://github.com/Fabian-Balzer/MfP2-Notizen}{GitHub}-Seite verfügbar. Zusätzlich benutzen wir das Lehrwerk Mathematik von Tilo Arens und Robins ausführliche Notizen aus 2019 als Quellen von guten Beispielen. Das Buch können wir jedem empfehlen, der auch Giancoli mag und am besten an Beispielen lernt. Bei Anmerkungen oder Fragen schreibt uns einfach auf Discord, Matrix/Element oder \href{https://github.com/luxdragon/MfP3-Notizen}{GitHub} an. \\ 

Möge die Macht der endlosen zerbrochenen Kreiden mit euch sein :)

\cfoot{\pagemark}
