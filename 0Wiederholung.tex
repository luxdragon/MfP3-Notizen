%%%%%%%%%%%%%%%%%%%%%%%%%%%%%%%%%%%%%%%%%%%%%%%%%%%%%%%%%%%%%%%%%
\newpage
\section[Wiederholung]{Riemann-Integrale und Untermannigfaltigkeiten}
\subsection{Riemann-Integrale}\label{ssec:Riemann-Integrale}
\begin{Def}
{Ober- und Unterintegral}
Sei $f:[a,b]\rightarrow \R$ eine beschränkte Funktion. Das \red{Oberintegral} von $f$ ist die Zahl
$$\int_a^{*b}f(x)dx=\inf \{ \int_a^b \varphi(x)dx | \varphi \text{ Treppenfunktion mit } \varphi\geq f \} $$
Das \red{Unterintegral} von $f$ ist die Zahl
$$\int_{*a}^{b}f(x)dx=\sup \{ \int_a^b \varphi(x)dx | \varphi \text{ Treppenfunktion mit } \varphi\leq f \} $$
\end{Def}

\begin{Def}
{Riemann-integrierbare Funktion}
Eine beschränkte Funktion $f:[a,b]\rightarrow \R$ heißt \red{Riemann-integrierbar}, wenn $$\int_a^{*b}f(x)dx=\int_{*a}^bf(x)dx$$
\end{Def}
Ebenso ist es in der Analysis nützlich den Begriff des uneigentlichen Integrals zu verstehen, wenn der Definitionsbereich unbeschränkt ist.
\begin{Def}
{Uneigentliches Integral}
Sei $f:(a,b]\rightarrow \R$ nicht unbedingt beschränkt, aber für jedes Teilinterval $[\alpha, b] \in (a,b]$ integrierbar. Falls dann der Grenzwert
$$\lim_{\epsilon\rightarrow a}\int_\epsilon^{b} f(x)dx$$ existiert, so nennen wir $f$ über das Intervall $(a,b]$ \red{uneigentlich integrierbar}.
\end{Def}
\begin{Beispiel}
{Uneigentliches Integral}
Betrachten wir $\int_0^1 \frac{1}{\sqrt{x}}dx$. Dieses Integral kann als uneigentliches integral sinnvoll definiert werden:
$$\lim_{\epsilon\rightarrow 0}\int_\epsilon^1 \frac{1}{\sqrt{x}} dx =\lim_{\epsilon\rightarrow 0} [2\sqrt{x}]_\epsilon^1=\lim_{\epsilon\rightarrow 0}2\sqrt{1}-2\sqrt{\epsilon}=2$$
\end{Beispiel}
Die Gamma-Funktion $\Gamma(s)=\int_0^\infty t^{s-1}e^{-t}dt$ ist sehr nützlich, denn es gilt $\Gamma(s+1)=s\Gamma(s)$, also ist die Funktion perfekt dazu geeignet, um die Fakultät zu berechnen: $\Gamma(n)=(n-1)!$. Zusätzlich gilt $\Gamma(\frac{1}{2})=\sqrt{\pi}$.

\subsection{Topologie und Untermannigfaltigkeiten}\label{ssec:Topologie}
Gewisse topologische Begriffe sind auch im MfP3 sehr wichtig.
\begin{Def}
{Offene Kugel}
Die Teilmenge $\boxed{B_r(\xvec):=\Menge{\yvec\in X}{d(\xvec,\yvec)<r}\subseteq X}$ eines metrischen Raumes $(X,d)$ mit Abstandsfunktion $d$ heißt \red{offene Kugel} mit Mittelpunkt $\xvec$ und Radius $r$.
\end{Def}
\begin{Def}
{Inneres}
Für eine Teilmenge $A\subseteq X$ nennen wir die Vereinigung aller offenen Teilmengen das \red{Innere $\mathring{A}$} von $A$, also $\mathring{A}=\bigcup_{B\subseteq A}B$.
\end{Def}

\begin{Def}
{Abschluss}
Für eine Teilmenge $A\subseteq X$ nennen wir den Durchschnitt aller abgeschlossenen Mengen, die $A$ enthalten, den \red{Abschluss $\Bar{A}$} von $A$, also $\Bar{A}=\bigcap_{B\supset A, B\tx{ abgeschl.}}B$.
\end{Def}
\begin{Def}
{Rand}
Die Differenz aus diesen beiden Mengen ist dann der \red{Rand $\partial A$} von $A$, also $\partial A=\Bar{A}\setminus \mathring{A}$.
\end{Def}
\begin{Def}
{Kompaktheit}
Falls wir zu \underline{jeder} offenen Überdeckung $(U_i)_{i\in I}$ von $A\subseteq X$ eine endliche Teilüberdeckung, d.h. eine Einschränkung der Indexmenge $I$ auf eine endliche Menge $J\subseteq I$ finden, sodass $(U_i)_{i\in J}$ immer noch eine offene Überdeckung von $A$ ist, so nennen wir $A\subseteq X$ \red{kompakt}.
\end{Def}
\begin{Satz}
{Folgerung}{Kompakte Teilmengen sind abgeschlossen und beschränkt}
Jede kompakte Teilmenge $A\subseteq X$ eines metr. Raumes ist \underline{abgeschlossen}, \underline{vollständig} und \underline{beschränkt}.
\end{Satz}
\begin{Def}{Separable Räume und Dichtigkeit}
    Die Metrik $(X,d)$ heißt separabel, falls es eine abzählbare dichte Teilmenge in $X$ gibt. \\
    Eine Teilmenge $Y\subseteq X$ heißt dicht in $X$, wenn $\overline{Y}=X$.
    $$\forall \epsilon>0 \forall x\in X \exists y\in Y: d(x,y) <\epsilon$$
\end{Def}
Jetzt gehen wir über zu Untermannigfaltigkeiten und schließen die Wiederholung mit allgemeinen Mannigfaltigkeiten ab.
\begin{Def}
{Immersion}
Ist $\rg(f)=m\,\forall \pvec\in U$, d. h. die Abbildung hat den konstanten Rang der Dimension des \underline{Urbildraums}, so sagen wir, dass $f$ eine \red{Immersion} ist.\\
Das Differential $df:U\to\mathbb{R}^n$ ist dann injektiv. 
\end{Def}
Die $C^k(I,\R)$ ist eine $k$-fach stetig differenzierbare Funktion.
\begin{Def}
{$C^k$-Diffeomorphismen}
$C^k$-Abbildungen $f:U\to V$, die bijektiv sind und deren Umkehrabbildungen auch $C^k$ sind, nennen wir \red{$C^k$-Diffeomorphismen}.
\end{Def}
\begin{Beispiel}
{Bekannter Diffeomorphismus}
Die Abbildung $f:\mathbb{R}\to\mathbb{R}_+\setminus\MengeDirekt{0},\,f(x)=e^x$ ist mit $f^{-1}:\mathbb{R}_+\setminus\MengeDirekt{0}\to\mathbb{R},\,f^{-1}(x)=\ln(x)$ ein Diffeomorphismus, denn beide sind stetig differenzierbar, bijektiv und\\
$(f\circ f^{-1}=\Id_{\mathbb{R}_+\setminus\MengeDirekt{0}})\land (f^{-1}\circ f=\Id_\mathbb{R}).\,\checkmark$
\end{Beispiel}
Eine wichtige Eigenschaft von Diffeomorphisem ist, dass sie homöomorph sind.
\begin{Def}
{Untermannigfaltigkeit}
Wir nennen eine Teilmenge $M\subseteq\mathbb{R}^n$ eine \red{$m$-dimensionale Untermannigfaltigkeit}, falls für jeden der Punkte $\pvec\in M$ die folgenden Eigenschaften erfüllt sind:
\begin{itemize}
    \item Es gibt eine offene Umgebung $V\subseteq\mathbb{R}^m$ von $\pvec$.
    \item Es gibt eine $C^k$-Immersion\footnote{also eine Abbildung von Rang $m$} $F:U\to\mathbb{R}^n$, die eine offene Teilmenge $U\subseteq\mathbb{R}^m$ homöomorph auf $V\cap M$ abbildet.
\end{itemize}
\end{Def}
\begin{Def}
{Flächen und Hyperflächen}
Zweidimensionale Untermannigfaltigkeiten des $\mathbb{R}^n$ nennen wir \red{Flächen}.\\
$(n-1)$-dimensionale Untermannigfaltigkeiten des $\mathbb{R}^n$ nennen wir \red{Hyperflächen}.
\end{Def}
\begin{Def}
{Lokale Parametrisierung}
Eine solche Abbildung (also $F:U\overset{\sim}{\to}F(U)\subseteq M$) nennen wir eine \red{lokale Parametrisierung} oder auch \red{Karte} der Umgebung $F(U)\subseteq M$ des Punktes $\pvec$.
\end{Def}
\begin{Beispiel}
    {Karte eines Funktionsgraphen}
    Sei $f: U\subseteq\mathbb{R}^n\rightarrow\mathbb{R}$ eine $C^k$-Funktion mit $k\geq 1$. Dann ist $F:U\subseteq\mathbb{R}^n\rightarrow\mathbb{R}^{n+1}, F(x)=(x,f(x))$ eine Immersion, denn
    \begin{equation*}
        \text{d}F_x=\Matrix{1 & \ldots & 0 \\ \vdots & \ddots & \vdots \\ 0 & \ldots & 1 \\ \partial_1 f(x) & \ldots & \partial_n f(x)} \quad \Rightarrow \quad \text{rg}(\text{d}F_x) = n
    \end{equation*}
    hat konstanten Rang $n$. Das Bild von $F$ ist der Graph $\Gamma_f$ von $f$. \\
    Es gibt offensichtlich eine Umkehrfunktion $F^{-1}:\,\Gamma_f\rightarrow\mathbb{R}\quad(x,f(x))\mapsto x$. Sowohl $F$ als auch $F^{-1}$ sind per Definition von $f$ stetig, sodass $F$ ein waschechter Homöomorphismus ist.
\end{Beispiel}
\begin{Satz}
{Satz}{Untermannigfaltigkeiten als Urbilder unter Abbildungen von konstantem Rang}
Sei $U\subseteq\mathbb{R}^n$ offen und $f:U\to\mathbb{R}^m$.
\begin{itemize}
    \item Ist $f$ eine $C^k$-Abbildung,
    \item hat $f$ konstanten Rang $r$ und
    \item ist $\qvec\in f(U)$,
\end{itemize}
so ist das Urbild\footnote{also alle Punkte, die auf $\qvec$ abgebildet werden} des Punktes $\qvec$ eine $C^k$-Untermannigfaltigkeit des $\mathbb{R}^n$ der Dimension \red{m=n-r}, also
\begin{equation}
    M:=f^{-1}(\qvec)\subseteq U.
\end{equation}
\end{Satz}
\begin{Beispiel}
    {Ein Beispiel von einer Untermannigfaltigkeit}
    Gegeben sind die beiden Funktionen
    \begin{equation*}
        f_1(x)=x_1^2+x_1x_2-x_2-x_3 \qquad f_2(x)=x_1^2+3x_1x_2-2x_2-3x_3
    \end{equation*}
    und die Menge
    \begin{equation*}
        C:=\Menge{x\in\mathbb{R}^3}{f_1(x)=f_2(x)=0}.
    \end{equation*}
    Wir behaupten nun, dass $C$ eine eindimensionale $C^{\infty}$-Untermannigfaltigkeit des $\mathbb{R}^3$ ist. Dazu betrachten wir die Funktion $f:\mathbb{R}^3\rightarrow\mathbb{R}^2$, gegeben durch
    \begin{equation*}
        f(x)=\Matrix{f_1(x) \\ f_2(x)}
    \end{equation*}
    und stellen fest:
    \begin{itemize}
        \item $f$ ist $\infty$-oft differenzierbar
        \item Das Differential ist
        \begin{align*}
            \text{d}f&=\Matrix{2x_1+x_2 & x_1-1 & -1 \\ 4x_1+3x_2 & 3x_1-2 & -3} \overset{Gauß}{\longrightarrow} \Matrix{2x_1 & x_1-1 & -1 \\ -2x_1 & -1 & 0} \\
            &\Rightarrow \text{rg}(f)=\text{rg}(\text{d}f)=2 \quad \forall x\in\mathbb{R}^3
        \end{align*}
        also ist $f$ eine Abbildung von konstantem Rang..
    \end{itemize}
    Die Behauptung folgt dann wieder direkt aus dem Satz über Untermannigfaltigkeiten als Urbilder unter Abbildungen von konstantem Rang. Die Untermannigfaltigkeit hat demnach auch die behauptete Dimension $3-2=1$.
\end{Beispiel}
\begin{Def}
{Tangentialvektoren und Tangentialraum}
Für eine $C^1$-Untermannigfaltigkeit $M\subseteq\mathbb{R}^n$ nennen wir für einen Punkt $\pvec\in M$ gelegte Vektoren $\vvec\in\mathbb{R}^n$ \red{Tangentialvektoren}, wenn
\begin{itemize}
    \item es $\epsilon>0$ und eine stetig differenzierbare Kurve $\gamma:(-\epsilon,\epsilon)\to M$ gibt, welche
    \begin{itemize}
        \item durch den Punkt $\pvec$ verläuft, also $\gamma(0)=\pvec$ und
        \item für die $\vvec=\gamma'(0)$ gilt.
    \end{itemize}
\end{itemize}
Die Menge aller Tangentialvektoren an die Untermannigfaltigkeit $M$ im Punkt $\pvec$ nennen wir \red{Tangentialraum} in $\pvec$.\\
Wir schreiben auch $T_\pvec M$.
\end{Def}
\begin{Satz}
{Satz}{Eigenschaften des Tangentialraums}
Für den Tangentialraum $T_\pvec M$ an einer UMF $M$ im Punkt $\pvec$ gilt:
\begin{itemize}
    \item $T_\pvec M$ ist ein Vektorraum mit $\dim (T_\pvec M)=m=\dim M$.
    \item Falls wir mit $F:U\to V$ eine lokale Parametrisierung von $M$ haben, sodass $F(\uvec)=\pvec$ ist ($\uvec\in U$),\\
    dann bilden die Vektoren $\partial_1F(\uvec),\ldots,\partial_mF(\uvec)$ eine Basis des Tangentialraums.
\end{itemize}
\end{Satz}
Den Tangentialraum können wir uns für einen Punkt $\pvec$ an einer Kugel $M$ als Ebene vorstellen, die an diesen Punkt angelegt wird.\\

Und nun zum krönenden Abschluss die allgemeinen Mannigfaltigkeiten.
\begin{Def}
    {Allgemeine Mannigfaltigkeiten}
    Gegeben seien ein metrischer Raum $M$, eine offene Überdeckung $(v_i)_{i\in I}$ von $M$ mit offenen Mengen $U_i \subseteq \mathbb{R}^m$ und Homöomorphismen 
    \begin{equation*}
        F_i:\, U_i\overset{\sim}{\rightarrow} V_i.
    \end{equation*}
    Man spricht dann von einer (abstrakten) m-dimensionalen $C^k$-Mannigfaltigkeit, wenn für je zwei offene Mengen $V_1,V_2\subseteq M$ mit Abbildungen $F_1$ und $F_2$ die Abbildung
    \begin{equation*}
        F_2^{-1}\circ F_1:\, F_1^{-1}(V_1\cap V_2) \rightarrow F_2^{-1}(V_1\cap V_2)
    \end{equation*}
    ein $C^k$-Diffeomorphismus ist.
\end{Def}